\section{MNIST手写数字识别}

\subsection{任务概述}

\subsubsection{MNIST 数据集}
MNIST 是手写数字识别经典数据集,包含 60000 张训练样本和 10000 张测试样本,每张样本为 28×28 像素的灰度图像,标签为 0-9 的数字,是深度学习入门的标准基准数据集。

\subsubsection{卷积神经网络(CNN)}
CNN 是专为处理网格结构数据(如图像)设计的深度学习模型,通过卷积层(Convolutional Layer)提取局部特征、池化层(Pooling Layer)降维并保留关键特征,全连接层(Fully Connected Layer)完成分类,相比全连接神经网络,CNN 具有参数少、抗过拟合、特征提取能力强的优势。

基于卷积神经网络(CNN)实现 MNIST 手写数字识别,具体任务步骤如下:
\begin{enumerate}[label=\arabic*.]
    \item \textbf{数据准备阶段}:导入深度学习所需核心库,设置随机种子保证实验可复现;对 MNIST 数据集进行预处理(如归一化、维度调整、标签编码等),并完成数据集的加载与划分(训练集/测试集)。
    \item \textbf{模型定义阶段}:构建轻量级卷积神经网络(CNN)模型,包含卷积层、池化层、全连接层等核心模块,实现手写数字特征提取与分类。
    \item \textbf{模型训练阶段}:选择合适的损失函数(如交叉熵损失)和优化器(如 Adam/SGD),设置训练超参数(批次大小、迭代轮数等),完成模型的训练过程,并监控训练过程中的损失/精度变化。
    \item \textbf{模型测试阶段}:使用测试集评估训练完成的模型性能,计算测试集的准确率、精确率等关键指标,验证模型泛化能力。
    \item \textbf{结果可视化阶段}:可视化训练/测试损失曲线、准确率曲线,以及随机抽取测试样本的预测结果(如展示手写数字图像与模型预测标签)。
\end{enumerate}


\subsubsection{核心技术栈}
\begin{itemize}
    \item 框架:PyTorch/TensorFlow/Keras(主流深度学习框架);
    \item 数据处理:NumPy、Pandas(数值计算),Matplotlib/Seaborn(可视化);
    \item 核心算法:卷积运算、池化操作、反向传播、梯度下降优化(Adam/SGD)。
\end{itemize}

\subsection{MNIST手写数字识别实现}
\subsubsection{Python 环境与框架搭建}
\begin{enumerate}[label=\arabic{enumi}.]
    \item \textbf{环境配置}:安装 Python 3.8+ 版本,通过 pip/conda 安装核心依赖包,命令如下:
    \begin{lstlisting}
# 安装PyTorch(以CPU版本为例,GPU版本需适配CUDA)
pip install torch torchvision
# 安装数据处理与可视化库
pip install numpy matplotlib pandas
    \end{lstlisting}
    \item \textbf{框架选择}:选用 PyTorch 框架实现,其动态计算图特性便于调试,且 torchvision 内置 MNIST 数据集加载接口,降低开发成本。
    \item \textbf{环境验证}:编写测试代码验证库是否安装成功,确保无导入错误。
\end{enumerate}

\subsubsection{数据集准备}
\begin{enumerate}[label=\arabic{enumi}.]
    \item \textbf{数据集加载}:通过 torchvision.datasets.MNIST 自动下载/加载数据集,指定数据存储路径、训练/测试集标识;
    \item \textbf{数据预处理}:
        \begin{itemize}
            \item 图像归一化:将像素值从 [0,255] 缩放到 [0,1],消除量纲影响;
            \item 维度调整:将 28×28 灰度图转换为 (1,28,28) 维度(通道数×高度×宽度),适配 CNN 输入格式;
            \item 数据增强(可选):如随机旋转、平移,提升模型泛化能力;
        \end{itemize}
    \item \textbf{数据加载器构建}:使用 torch.utils.data.DataLoader 封装数据集,设置批次大小(\texttt{batch\_size})、是否打乱(\texttt{shuffle})等参数,实现批量数据读取。
\end{enumerate}

\subsubsection{Python 代码实现和调试}
\lstset{
    language=Python,
    basicstyle=\small\ttfamily,
    keywordstyle=\color{blue},
    commentstyle=\color{gray},
    stringstyle=\color{red},
    frame=single,
    breaklines=true,
    showspaces=false,
    showstringspaces=false,
    numbers=left,          % 显示行号
    numberstyle=\tiny\color{gray}, % 行号样式
    captionpos=b           % 代码标题在下方
}
\begin{enumerate}
  \item 环境初始化与数据预处理:设置随机种子保证实验可复现性,定义训练超参数(轮数、批次大小、学习率),通过torchvision加载MNIST数据集并完成归一化预处理,构建数据加载器实现批量训练。
  \begin{lstlisting}[caption={环境初始化与数据加载}]
import torch
import torch.nn as nn
import torch.utils.data as Data
import torchvision
import matplotlib.pyplot as plt
import numpy as np

# 设置随机种子确保可复现性
torch.manual_seed(1)

# 超参数定义
EPOCH = 3  # 训练轮数
BATCH_SIZE = 50
LR = 0.001

# 加载MNIST数据集并归一化(ToTensor自动将像素值缩放到[0,1])
train_data = torchvision.datasets.MNIST(
    root='./data/', train=True, transform=torchvision.transforms.ToTensor(), download=True
)
test_data = torchvision.datasets.MNIST(
    root='./data/', train=False, transform=torchvision.transforms.ToTensor()
)

# 构建批训练加载器,shuffle=True打乱训练数据
train_loader = Data.DataLoader(dataset=train_data, batch_size=BATCH_SIZE, shuffle=True)

# 测试集预处理(适配CNN输入维度)
test_x = test_data.data[:2000].unsqueeze(1).type(torch.FloatTensor) / 255.0
test_y = test_data.targets[:2000]
\end{lstlisting}
\item 轻量级CNN模型定义:构建包含两层卷积+池化、一层全连接的轻量级CNN:卷积层提取图像局部特征,池化层降维减少参数,全连接层完成10分类(0-9数字);forward函数定义前向传播流程。
\begin{lstlisting}[caption={CNN模型定义}]
class CNN(nn.Module):
    def __init__(self):
        super(CNN, self).__init__()
        # 第一层卷积:1通道→16通道,卷积核5×5,填充2,ReLU激活+最大池化
        self.conv1 = nn.Sequential(
            nn.Conv2d(1, 16, 5, 1, 2),  # 输入[batch,1,28,28] → 输出[batch,16,28,28]
            nn.ReLU(),
            nn.MaxPool2d(2),             # 池化后[batch,16,14,14]
        )
        # 第二层卷积:16通道→32通道,同上
        self.conv2 = nn.Sequential(
            nn.Conv2d(16, 32, 5, 1, 2),  # 输入[batch,16,14,14] → 输出[batch,32,14,14]
            nn.ReLU(),
            nn.MaxPool2d(2),             # 池化后[batch,32,7,7]
        )
        # 全连接层:展平特征→10分类输出
        self.out = nn.Linear(32 * 7 * 7, 10)

    def forward(self, x):
        x = self.conv1(x)
        x = self.conv2(x)
        x = x.view(x.size(0), -1)  # 展平:[batch,32,7,7] → [batch,32×7×7]
        output = self.out(x)
        return output

# 初始化模型
cnn = CNN()
\end{lstlisting}
\item 模型训练与指标记录:选择Adam优化器和交叉熵损失函数,迭代训练模型;每50步计算测试集准确率,记录训练损失和准确率,用于后续可视化。
\begin{lstlisting}[caption={模型训练}]
# 优化器和损失函数
optimizer = torch.optim.Adam(cnn.parameters(), lr=LR)
loss_func = nn.CrossEntropyLoss()

# 初始化训练指标记录字典
train_metrics = {'epochs': [], 'steps': [], 'losses': [], 'accuracies': []}

# 开始训练
for epoch in range(EPOCH):
    for step, (b_x, b_y) in enumerate(train_loader):
        output = cnn(b_x)          # 前向传播
        loss = loss_func(output, b_y)  # 计算损失

        optimizer.zero_grad()      # 梯度清零
        loss.backward()            # 反向传播
        optimizer.step()           # 更新参数

        # 每50步记录并打印训练指标
        if step % 50 == 0:
            test_output = cnn(test_x)
            pred_y = torch.max(test_output, 1)[1].numpy()
            # 计算测试集准确率(百分比)
            accuracy = float((pred_y == test_y.numpy()).sum()) / float(test_y.size(0)) * 100
            
            # 记录指标
            train_metrics['epochs'].append(epoch)
            train_metrics['steps'].append(step)
            train_metrics['losses'].append(loss.item())
            train_metrics['accuracies'].append(accuracy)
            
            # 打印训练状态
            print(f'Epoch: {epoch}/{EPOCH} | Step: {step} | Loss: {loss.item():.4f} | Test Acc: {accuracy:.2f}%')

# 保存训练好的模型
torch.save(cnn.state_dict(), 'cnn_mnist.pkl')
\end{lstlisting}
\item 结果可视化:使用Matplotlib绘制训练损失曲线和测试准确率曲线,直观展示模型训练过程中的性能变化。
\begin{lstlisting}[caption={结果可视化(核心片段)}]
# 1. 绘制损失&准确率曲线
plt.figure(figsize=(10, 5))
# 子图1:损失曲线
plt.subplot(1, 2, 1)
plt.plot(train_metrics['steps'], train_metrics['losses'], 'b-', label='Training Loss')
plt.xlabel('Steps'), plt.ylabel('Loss'), plt.title('Training Loss Curve'), plt.grid(True)

# 子图2:准确率曲线
plt.subplot(1, 2, 2)
plt.plot(train_metrics['steps'], train_metrics['accuracies'], 'r-', label='Test Accuracy')
plt.xlabel('Steps'), plt.ylabel('Accuracy (%)'), plt.title('Test Accuracy Curve'), plt.ylim(0,100), plt.grid(True)

plt.tight_layout(), plt.savefig('training_metrics.png'), plt.show()

# 2. 可视化32个样本预测结果(区分正确/错误)
inputs = test_x[:32]
test_output = cnn(inputs)
pred_y = torch.max(test_output, 1)[1].numpy()
true_y = test_y[:32].numpy()

plt.figure(figsize=(16, 8))
for i in range(32):
    plt.subplot(4, 8, i+1)
    img = inputs[i].squeeze().numpy()
    plt.imshow(img, cmap='gray')
    # 正确标绿,错误标红
    color = 'green' if pred_y[i]==true_y[i] else 'red'
    plt.title(f'True: {true_y[i]}\nPred: {pred_y[i]}', color=color, fontsize=8)
    plt.xticks([]), plt.yticks([])

plt.suptitle(f'Prediction Results (Acc: {(pred_y==true_y).sum()/32*100:.1f}%)'), plt.tight_layout(), plt.savefig('predictions.png'), plt.show()

# 3. 错误样本单独可视化
wrong_idx = np.where(pred_y != true_y)[0]
if len(wrong_idx) > 0:
    plt.figure(figsize=(12, 6))
    for i, idx in enumerate(wrong_idx[:16]):  # 最多展示16个错误样本
        plt.subplot(4, 4, i+1)
        img = inputs[idx].squeeze().numpy()
        plt.imshow(img, cmap='gray')
        plt.title(f'True: {true_y[idx]}\nPred: {pred_y[idx]}', color='red', fontsize=10)
        plt.xticks([]), plt.yticks([])
    plt.suptitle(f'Wrong Predictions (Error Rate: {len(wrong_idx)/32*100:.2f}%)'), plt.tight_layout(), plt.savefig('wrong_predictions.png'), plt.show()
\end{lstlisting}

\end{enumerate}

\newpage
\subsection{测试结果与分析}

\begin{enumerate}
  \item \textbf{测试结果}:经过3轮训练,模型在2000个测试样本上达到约98.5\%的准确率,损失值逐步下降,表明模型有效学习到了手写数字的特征。
  \begin{figure}[htbp]
    \centering
    \includegraphics[width=0.8\textwidth]{images/mnist_predictions_with_labels.png}
    \caption{部分测试样本预测结果可视化}
  \end{figure}
  \item 损失曲线显示训练过程中损失稳步下降,准确率曲线则显示测试准确率逐渐提升,验证了模型的收敛性和泛化能力。
  \begin{figure}[htbp]
    \centering
    \includegraphics[width=0.8\textwidth]{images/training_accuracy_curve.png}
    \caption{训练损失与测试准确率曲线训练损失与测试准确率曲线}
  \end{figure}
  \item 可视化结果中,大部分样本预测正确,错误样本数量较少,且错误样本多为书写模糊或形似数字,符合预期。
\end{enumerate}

\subsection{总结与收获}
本次基于CNN实现MNIST手写数字识别的实践,从环境搭建、数据处理到模型训练与可视化,完整覆盖了深度学习入门的核心流程,主要总结与收获如下:

\subsubsection{1. 技术能力提升}
\begin{enumerate}[ itemsep=5pt]
    \item \textbf{深度学习基础落地能力}:掌握了PyTorch框架下数据集加载、预处理的标准流程,理解了ToTensor归一化、维度调整等操作对CNN输入的适配逻辑;掌握了CNN核心模块(卷积层、池化层、全连接层)的设计原理,能根据输入输出维度合理设计网络结构,理解“特征提取-降维-分类”的核心范式。
    \item \textbf{模型训练与调优思维}:熟悉了Adam优化器、交叉熵损失函数的适用场景,掌握了训练循环的编写逻辑(前向传播→损失计算→反向传播→参数更新);学会通过监控训练损失和测试准确率判断模型训练状态,理解超参数(学习率、批次大小、训练轮数)对模型性能的影响。
    \item \textbf{结果可视化与分析能力}:掌握了Matplotlib绘制训练曲线、样本预测结果的方法,能通过可视化直观分析模型训练趋势(如损失下降、准确率提升),并针对错误样本定位模型不足,形成“训练-评估-分析”的闭环思维。
\end{enumerate}

\subsubsection{2. 工程实践认知}
\begin{enumerate}[ itemsep=5pt]
    \item \textbf{可复现性与规范性}:理解了设置随机种子的重要性,保证实验结果可复现;学会模块化编写代码(数据加载、模型定义、训练、可视化分离),提升代码可读性和可维护性。
    \item \textbf{问题排查与调试}:在实践中解决了数据集维度不匹配、测试集加载兼容等问题,掌握了通过打印维度、分步验证的方式定位代码错误,提升了调试能力。
    \item \textbf{模型性能认知}:通过本次实践验证了轻量级CNN在MNIST数据集上的优异性能(测试准确率可达98\%以上),理解了CNN相比全连接网络在图像识别任务中的优势(参数更少、特征提取更高效)。
\end{enumerate}

\subsubsection{3. 后续改进方向}
本次实践为深度学习入门打下了基础,后续可从以下方向优化:
\begin{itemize}[itemsep=3pt]
    \item 引入数据增强(如随机旋转、平移、缩放),进一步提升模型泛化能力;
    \item 尝试调整网络结构(如增加卷积层、使用Dropout层),解决潜在的过拟合问题;
    \item 对比不同优化器(SGD、RMSprop)、学习率调度策略对模型训练的影响;
    \item 部署训练好的模型,实现单张手写数字图片的离线预测功能。
\end{itemize}

\subsubsection{4. 核心感悟}
深度学习的核心是“数据+模型+训练”的协同优化,本次实践让我深刻体会到:入门阶段无需追求复杂模型,通过经典数据集(MNIST)掌握基础流程和核心原理,是理解深度学习的关键;而可视化分析和问题调试能力,是从“代码实现”到“理解本质”的重要桥梁。本次实践不仅掌握了具体的技术方法,更建立了深度学习的工程化思维,为后续更复杂的图像识别任务奠定了基础。